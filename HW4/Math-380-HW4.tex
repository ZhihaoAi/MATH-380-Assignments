\documentclass[10pt]{report}

\usepackage{geometry}
\geometry{
	letterpaper,
	hmargin=0.5in,
	vmargin=1in,
	footskip=0.25in
}

\usepackage{enumerate} % for enumerate counter
\usepackage{subcaption} % for subfigures
\usepackage{amsthm} % for QED
\usepackage{mathtools} % for delimiter

\usepackage{listings} % for code
\lstset{ 
	language=R,
	basicstyle=\footnotesize\ttfamily,
	numbers=none,
	stepnumber=1,
	numbersep=8pt,
	showspaces=false,
	showstringspaces=false,
	showtabs=false,
	frame=single,
	tabsize=2,
	captionpos=t,
	breaklines=true,
	breakatwhitespace=false
} 

\usepackage{float} % for figure [H]
\usepackage{booktabs} % for tabular
\usepackage{caption} % for \caption*
\usepackage[export]{adjustbox} % for valign=t
\usepackage{array} % for column type m
\usepackage{verbatim}
\usepackage{graphicx}
%\graphicspath{ {imgs/} }

\usepackage{fancyhdr}
\pagestyle{fancy}
\fancyhead[L]{\hwAuther}
\fancyhead[C]{\courseNo}
\fancyhead[R]{\hwNo}

\usepackage{amssymb}
\usepackage{amsmath}

%Cover
\newcommand{\courseTitle}{Introduction to Mathematical Modeling}
\newcommand{\courseNo}{Math 380}
\newcommand{\hwAuther}{Zhihao Ai}

\newcommand{\hwNo}{HW \#4}
\newcommand{\hwDate}{Due on 02/20}

\title{
	\courseTitle\\
	\hwNo\\
	\hwDate
}
\author{\hwAuther}
\date{}
%

%Custom
%\everymath{\displaystyle}
\setlength\parindent{0pt}

%Custom commands
\newcommand{\ds}{\displaystyle}
\newcommand{\ts}{\textstyle}

\newcolumntype{N}{>$ c <$} 
\newcolumntype{M}[1]{>{\centering\arraybackslash $}m{#1}<{$}}

\newcommand{\abs}[1] {\left| #1 \right|}

\DeclarePairedDelimiter\autoparen{(}{)}
\newcommand{\pa}[1]{\autoparen*{#1}}

\newcommand{\var} {\text{var}}

\newcommand{\m}[1] {\mathbf{#1}}

\begin{document}

\maketitle

\section*{Section 3.3}
\begin{enumerate}
	\item [4.]
	Make an appropriate transformation to fit the model $P=ae^{bt}$ using Equation (3.4). Estimate $a$ and $b$.
	\begin{table}[H]
		\centering
		\begin{tabular}{*{7}{c}} 
			\toprule
			$t$ & 7 & 14 & 21 & 28 & 35 & 42 \\ \midrule
			$P$ & 8 & 41 & 133 & 250 & 280 & 297 \\
			\bottomrule
		\end{tabular}
	\end{table}
	Let $y=\ln{P}$, $ae^{bt} = \ln{a} + bt$, $A=\ln{a}$, we have
	\begin{table}[H]
		\centering
		\begin{tabular}{*{7}{c}} 
			\toprule
			$t$ & 7 & 14 & 21 & 28 & 35 & 42 \\ \midrule
			$y$ & 2.08 & 3.71 & 4.89 & 5.52 & 5.63 & 5.69 \\
			\midrule
			$t^2$ & 49 & 196 & 441 & 784 & 1225 & 1764 \\
			\midrule
			$t y$ & 14.56 & 51.99 & 102.70 & 154.60 & 197.22 & 239.14 \\
			\bottomrule
		\end{tabular}
	\end{table}
	Using Equation (3.4), we have
	\begin{align*}
		4459b + 147A &= 760.2\\
		147A + 6b &= 27.53
	\end{align*}
	Solving the equations, $A=2.14$, $b=0.1$, so $a = e^A = 8.5$.

	\item [8.]
	Fit the data with the models given, using least squares.
	\begin{table}[H]
		\centering
		\begin{tabular}{*{10}{c}} 
			\toprule
			$x(\times 10^{-3})$ & 5 & 10 & 20 & 30 & 40 & 50 & 60 & 70 & 80 \\ \midrule
			$y(\times 10^{-5})$ & 0 & 19 & 57 & 94 & 134 & 173 & 216 & 256 & 297 \\
			\bottomrule
		\end{tabular}
	\end{table}
	\begin{enumerate}[a.]
		\item 
		$y=ax$
		\begin{align*}
			a \sum_{i=1}^{9} x_i^2 &= \sum_{i=1}^{9} x_i y_i\\
			0.0204a &= 0.000728\\
			a &= 0.0356
		\end{align*}
		So, $y=0.0356x$.
		
		\item 
		$y=b+ax$
		
		Solving
		\begin{align*}
			0.020425a + 0.365b &= 0.000728\\
			0.365a + 9b &= 0.01246
		\end{align*}
		we have $a=0.0396$ and $b=-0.000222$. So, $y=0.0396x - 0.000222$.
	\end{enumerate}
\end{enumerate}

\section*{Section 3.4}
\begin{enumerate}
	\item [7.]
	\begin{enumerate}
		\item [a.]
		In the following data, $W$ represents the weight of a fish and $l$ represents its length. Fit the model $W=kl^3$ to the data using the least-squares criterion.
		\begin{table}[H]
			\centering
			\begin{tabular}{*{9}{c}} 
				\toprule
				Length, $l$ (in.) & 14.5 & 12.5 & 17.25 & 14.5 & 12.625 & 17.75 & 14.125 & 12.625 \\ \midrule
				Weight, $W$ (oz) & 27 & 17 & 41 & 26 & 17 & 49 & 23 & 16  \\
				\bottomrule
			\end{tabular}
		\end{table}
		Sum of square error
		\[
		S = \sum_{i=1}^{8} (W_i - kl_i^3)^2
		\]
		Setting the derivative to 0, we have
		\begin{align*}
			\frac{\partial S}{\partial k} &= 0\\
			-2 \sum_{i=1}^{8} (W_i - kl_i^3)l_i^3 &= 0\\
			k &= \frac{\sum_{i=1}^{8} W_i l_i^3}{\sum_{i=1}^{8} l_i^6}\\
			k &= 0.00844
		\end{align*}
		Thus, $W=0.00844l^3$.
		
		\item [b.]
		In the following data, $g$ represents the girth of a fish. Fit the model $W=klg^2$ to the data using the least-squares criterion.
		\begin{table}[H]
			\centering
			\begin{tabular}{*{9}{c}} 
				\toprule
				Length, $l$ (in.) & 14.5 & 12.5 & 17.25 & 14.5 & 12.625 & 17.75 & 14.125 & 12.625 \\ \midrule
				Girth, $g$ (in.) & 9.75 & 8.375 & 11.0 & 9.75 & 8.5 & 12.5 & 9.0 & 8.5  \\ \midrule
				Weight, $W$ (oz) & 27 & 17 & 41 & 26 & 17 & 49 & 23 & 16  \\
				\bottomrule
			\end{tabular}
		\end{table}
		Sum of square error
		\[
		S = \sum_{i=1}^{8} (W_i - k l_i g_i^2)^2
		\]
		Setting the derivative to 0, we have
		\begin{align*}
			\frac{\partial S}{\partial k} &= 0\\
			-2 \sum_{i=1}^{8} (W_i - k l_i g_i^2) l_i g_i^2 &= 0\\
			k &= \frac{\sum_{i=1}^{8} W_i l_i g_i^2}{\sum_{i=1}^{8} l_i^2 g_i^4}\\
			k &= 0.0187
		\end{align*}
		Thus, $W=0.0187lg^2$.
	\end{enumerate}
	
	\item [8.]
	Use the data presented in Problem 7b to fit the models $W=cg^3$ and $W=kgl^2$. Interpret these models. Compute appropriate indicators and determine which model is best. Explain.
	
	To fit $W=cg^3$, setting the derivative to 0, we have
	\begin{align*}
		\frac{\partial S}{\partial c} &= 0\\
		-2 \sum_{i=1}^{8} (W_i - c g_i^3) g_i^3 &= 0\\
		c &= \frac{\sum_{i=1}^{8} W_i g_i^3}{\sum_{i=1}^{8} g_i^6}\\
		c &= 0.0276
	\end{align*}
	Thus, $W=0.0276g^3$.
	
	To fit $W=kgl^2$, setting the derivative to 0, we have
	\begin{align*}
		\frac{\partial S}{\partial k} &= 0\\
		-2 \sum_{i=1}^{8} (W_i - k l_i^2 g_i) l_i^2 g_i &= 0\\
		k &= \frac{\sum_{i=1}^{8} W_i l_i^2 g_i}{\sum_{i=1}^{8} l_i^4 g_i^2}\\
		k &= 0.0126
	\end{align*}
	Thus, $W=0.0126gl^2$.
	
	The summary of the results for all 4 models is shown below:
	\begin{table}[H]
		\centering
		\begin{tabular}{*{3}{c}} 
			\toprule
			Model & $\sum [W_i - f(l_i, g_i)]^2$ & $\max \abs{W_i - f(l_i, g_i)} $ \\ \midrule
			$W=0.00844l^3$ & 12.1693 & 2.32212  \\ \midrule
			$W=0.0187lg^2$ & 17.6832 & 2.86328  \\ \midrule
			$W=0.0276g^3$ & 54.2605 & 4.90625  \\ \midrule
			$W=0.0126gl^2$ & 3.39977 & 1.17079  \\ 
			\bottomrule
		\end{tabular}
	\end{table}
	Using either least square or Chebyshev criterion, $W=0.0126gl^2$ is the best model.
\end{enumerate}

\end{document}

