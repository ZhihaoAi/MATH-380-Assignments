\documentclass[10pt]{report}

\usepackage{geometry}
\geometry{
	letterpaper,
	hmargin=0.7in,
	vmargin=1in,
	footskip=0.25in
}

\usepackage{enumerate} % for enumerate counter
\usepackage{subcaption} % for subfigures
\usepackage{amsthm} % for QED
\usepackage{mathtools} % for delimiter

\usepackage{listings} % for code
\lstset{ 
	language=R,
	basicstyle=\footnotesize\ttfamily,
	numbers=none,
	stepnumber=1,
	numbersep=8pt,
	showspaces=false,
	showstringspaces=false,
	showtabs=false,
	frame=single,
	tabsize=2,
	captionpos=t,
	breaklines=true,
	breakatwhitespace=false
} 

\usepackage{float} % for figure [H]
\usepackage{booktabs} % for tabular
\usepackage{caption} % for \caption*
\usepackage[export]{adjustbox} % for valign=t
\usepackage{array} % for column type m
\usepackage{verbatim}
\usepackage{graphicx}
%\graphicspath{ {imgs/} }

\usepackage{fancyhdr}
\pagestyle{fancy}
\fancyhead[L]{\hwAuther}
\fancyhead[C]{\courseNo}
\fancyhead[R]{\hwNo}

\usepackage{amssymb}
\usepackage{amsmath}

%Cover
\newcommand{\courseTitle}{Introduction to Mathematical Modeling}
\newcommand{\courseNo}{Math 380}
\newcommand{\hwAuther}{Zhihao Ai}

\newcommand{\hwNo}{HW \#8}
\newcommand{\hwDate}{Due on 04/03}

\title{
	\courseTitle\\
	\hwNo\\
	\hwDate
}
\author{\hwAuther}
\date{}
%

%Custom
%\everymath{\displaystyle}
\setlength\parindent{0pt}

%Custom commands
\newcommand{\ds}{\displaystyle}
\newcommand{\ts}{\textstyle}

\newcolumntype{N}{>$ c <$} 
\newcolumntype{M}[1]{>{\centering\arraybackslash $}m{#1}<{$}}

\newcommand{\abs}[1] {\left| #1 \right|}

\DeclarePairedDelimiter\autoparen{(}{)}
\newcommand{\pa}[1]{\autoparen*{#1}}

\newcommand{\var} {\text{var}}

\newcommand{\m}[1] {\mathbf{#1}}

\begin{document}

\maketitle

\begin{enumerate}
	\item 
	Input: $n=1,000,000$ random points to be generated\\
	Output: Estimate of $\pi$
	\begin{enumerate}[i.]
		\item
		Define the bounding box as $S: 0 \le x \le 1$ and $0 \le y \le 1$
		
		\item
		Initialize $c=0$
		
		\item
		For $i=1, \dots, n$, do steps 3a-b
		\begin{enumerate}[a.]
			\item 
			Generate random points $x_i$, $y_i$, where $0 \le x_i \le 1$ and $0 \le y_i \le 1$
			
			\item 
			If $x_i^2 + y_i^2 \le 1$, then $c=c+1$
		\end{enumerate}
	
		\item
		$\hat{\pi} = 4c/n$
	\end{enumerate}
	Running the algorithm, we have $\hat{\pi} = 3.138052$.
	
	\item 
	Input: $n=1,000,000$ random points to be generated\\
	Output: Area trapped between $f_1(x) = x^2$ and $f_2(x) = 6-x$ and the $x$- and $y$-axes
	\begin{enumerate}[i.]
		\item
		Define the bounding region as $S: 0 \le x \le 6$ and $0 \le y \le 6$
		
		\item
		Initialize $c=0$
		
		\item
		For $i=1, \dots, n$, do steps 3a-b
		\begin{enumerate}[a.]
			\item 
			Generate random points $x_i$, $y_i$, where $0 \le x_i \le 6$ and $0 \le y_i \le 6$
			
			\item 
			If $y_i \le \min\{f_1(x_i), f_2(x_i)\}$, then $c=c+1$
		\end{enumerate}
		
		\item
		Area $ = 36 c/n$
	\end{enumerate}
	Running the algorithm, we have the estimated area $ = 10.656504$.
	
	\item
	Input: $n=1,000,000$ random points to be generated\\
	Output: Volume trapped between $f_1(x, y) = 8 - x^2 - y^2$ and $f_2(x, y) = x^2 + 3y^2$
	\begin{enumerate}[i.]
		\item
		Since the maximum value of $z=8-x^2-y^2$ is 8 and the minimum value of $z = x^2 + 3y^2$ is 0, and the two paraboloids intersect on $x^2 + 2y^2 = 4$, we can define the bounding volume as $S: -2 \le x \le 2$, $-2 \le y \le 2$ and $0 \le z \le 8$
		
		\item
		Initialize $c=0$
		
		\item
		For $i=1, \dots, n$, do steps 3a-b
		\begin{enumerate}[a.]
			\item 
			Generate random points $x_i$, $y_i$, $z_i$, where $-2 \le x \le 2$, $-2 \le y \le 2$ and $0 \le y_i \le 8$
			
			\item 
			If $z_i \le \min\{f_1(x_i, y_i), f_2(x_i, y_i)\}$, then $c=c+1$
		\end{enumerate}
		
		\item
		Estimated volume $ =  4\cdot 4\cdot 8\cdot c/n$
	\end{enumerate}
	Running the algorithm, we have the estimated volume $ = 35.505664$. The true value is $ \int_{-\sqrt{2}}^{\sqrt{2}} \int_{-\sqrt{4-2y^2}}^{\sqrt{4-2y^2}} (8 - x^2 - y^2) - (x^2 + 3y^2) dx dy = 35.5431$. They are fairly close.
	
	\item 
	Using the middle-square method, we have
	\begin{table}[H]
		\centering
		\begin{minipage}{.3\linewidth}
			\centering
			\begin{tabular}{*{3}{c}} 
				\toprule
				$i$ & $x_i$ & $x_i^2$ \\ \midrule
				0 & 653217 & 426692449089 \\ \midrule
				1 & 692449 & 479485617601 \\ \midrule
				2 & 485617 & 235823870689 \\ \midrule
				3 & 823870 & 678761776900 \\ \midrule
				4 & 761776 & 580302674176 \\ \midrule
				5 & 302674 & 91611550276 \\ \midrule
				6 & 611550 & 373993402500 \\
				\bottomrule
			\end{tabular}
		\end{minipage}
		\hspace{-2ex}
		\begin{minipage}{.3\linewidth}
			\centering
			\begin{tabular}{*{3}{c}} 
				\toprule
				$i$ & $x_i$ & $x_i^2$ \\ \midrule
				7 & 993402 & 986847533604 \\ \midrule
				8 & 847533 & 718312186089 \\ \midrule
				9 & 312186 & 97460098596 \\ \midrule
				10 & 460098 & 211690169604 \\ \midrule
				11 & 690169 & 476333248561 \\ \midrule
				12 & 333248 & 111054229504 \\ \midrule
				13 & 54229 & 2940784441 \\ 
				\bottomrule
			\end{tabular}
		\end{minipage}
		\hspace{-2ex}
		\begin{minipage}{.3\linewidth}
			\centering
			\begin{tabular}{*{3}{c}} 
				\toprule
				$i$ & $x_i$ & $x_i^2$ \\ \midrule
				14 & 940784 & 885074534656 \\ \midrule
				15 & 74534 & 5555317156 \\ \midrule
				16 & 555317 & 308376970489 \\ \midrule
				17 & 376970 & 142106380900 \\ \midrule
				18 & 106380 & 11316704400 \\ \midrule
				19 & 316704 & 100301423616 \\ \midrule
				20 & 301423 & 90855824929 \\ 
				\bottomrule
			\end{tabular}
		\end{minipage}
	\end{table}
	
	\item 
	Let $x_0=2$ then we have
	\begin{table}[H]
		\centering
		\begin{minipage}{.3\linewidth}
			\centering
			\begin{tabular}{*{3}{c}} 
				\toprule
				$i$ & $x_i$ & $(5 \cdot x_i + 3) \mod 16$ \\ \midrule
				0 & 2 & 13 \\ \midrule
				1 & 13 & 4 \\ \midrule
				2 & 4 & 7 \\ \midrule
				3 & 7 & 6 \\ \midrule
				4 & 6 & 1 \\ \midrule
				5 & 1 & 8 \\ \midrule
				6 & 8 & 11 \\ 
				\bottomrule
			\end{tabular}
		\end{minipage}
		\hspace{-2ex}
		\begin{minipage}{.3\linewidth}
			\centering
			\begin{tabular}{*{3}{c}} 
				\toprule
				$i$ & $x_i$ & $(5 \cdot x_i + 3) \mod 16$ \\ \midrule
				7 & 11 & 10 \\ \midrule
				8 & 10 & 5 \\ \midrule
				9 & 5 & 12 \\ \midrule
				10 & 12 & 15 \\ \midrule
				11 & 15 & 14 \\ \midrule
				12 & 14 & 9 \\ \midrule
				13 & 9 & 0 \\ 
				\bottomrule
			\end{tabular}
		\end{minipage}
		\hspace{-2ex}
		\begin{minipage}{.3\linewidth}
			\centering
			\begin{tabular}{*{3}{c}} 
				\toprule
				$i$ & $x_i$ & $(5 \cdot x_i + 3) \mod 16$ \\ \midrule
				14 & 0 & 3 \\ \midrule
				15 & 3 & 2 \\ \midrule
				16 & 2 & 13 \\ \midrule
				17 & 13 & 4 \\ \midrule
				18 & 4 & 7 \\ \midrule
				19 & 7 & 6 \\ \midrule
				20 & 6 & 1 \\ 
				\bottomrule
			\end{tabular}
		\end{minipage}
	\end{table}
\end{enumerate}

\end{document}

