\documentclass[10pt]{report}

\usepackage{geometry}
\geometry{
	a4paper,
	margin=1in,
	footskip=0.25in
}

\usepackage{enumerate} % for enumerate counter
\usepackage{subcaption} % for subfigures
\usepackage{amsthm} % for QED
\usepackage{mathtools} % for delimiter

\usepackage{listings} % for code
\lstset{ 
	language=R,
	basicstyle=\footnotesize\ttfamily,
	numbers=none,
	stepnumber=1,
	numbersep=8pt,
	showspaces=false,
	showstringspaces=false,
	showtabs=false,
	frame=single,
	tabsize=2,
	captionpos=t,
	breaklines=true,
	breakatwhitespace=false
} 

\usepackage{float} % for figure [H]
\usepackage{booktabs} % for tabular
\usepackage{caption} % for \caption*
\usepackage[export]{adjustbox} % for valign=t
\usepackage{array} % for column type m
\usepackage{verbatim}
\usepackage{graphicx}
%\graphicspath{ {imgs/} }

\usepackage{fancyhdr}
\pagestyle{fancy}
\fancyhead[L]{\hwAuther}
\fancyhead[C]{\courseNo}
\fancyhead[R]{\hwNo}

\usepackage{amssymb}
\usepackage{amsmath}

%Cover
\newcommand{\courseTitle}{Introduction to Mathematical Modeling}
\newcommand{\courseNo}{Math 380}
\newcommand{\hwAuther}{Zhihao Ai}

\newcommand{\hwNo}{HW \#3}
\newcommand{\hwDate}{Due on 02/13}

\title{
	\courseTitle\\
	\hwNo\\
	\hwDate
}
\author{\hwAuther}
\date{}
%

%Custom
%\everymath{\displaystyle}
\setlength\parindent{0pt}

%Custom commands
\newcommand{\ds}{\displaystyle}
\newcommand{\ts}{\textstyle}

\newcolumntype{N}{>$ c <$} 
\newcolumntype{M}[1]{>{\centering\arraybackslash $}m{#1}<{$}}

\newcommand{\abs}[1] {\left| #1 \right|}

\DeclarePairedDelimiter\autoparen{(}{)}
\newcommand{\pa}[1]{\autoparen*{#1}}

\newcommand{\var} {\text{var}}

\newcommand{\m}[1] {\mathbf{#1}}

\begin{document}

\maketitle

\section*{Section 3.1}
\begin{enumerate}
	\item [5.]
	The following data represent the growth of a population of fruit flies over a 6-week period. Test the following models by plotting an appropriate set of data. Estimate the parameters of the following models.
	\begin{table}[H]
		\centering
		\begin{tabular}{*{7}{c}} 
			\toprule
			$t$ (days) & 7 & 14 & 21 & 28 & 35 & 42 \\ \midrule
			$P$ (number of observed flies) & 8 & 41 & 133 & 250 & 280 & 297 \\
			\bottomrule
		\end{tabular}
	\end{table}
	\begin{enumerate}[a.]
		\item 
		$P = c_1 t$
		
		\item 
		$P = a e^{bt}$
	\end{enumerate}

	\item [7.]
	In 1601 the German astronomer Johannes Kepler became director of the Prague Observatory. Kepler had been helping Tycho Brahe in collecting 13 years of ovservations on the relative motion of the planet Mars. By 1609 Kepler had formulated his first two laws:
	\begin{enumerate}[i.]
		\item 
		Each planet moves on an ellipse with the sun at one focus.
		
		\item 
		For each planet, the line from the sun to the planet sweeps out equal areas in equal times.
	\end{enumerate}
	Kepler spent many years verifying these laws and formulating a third law, which relates the planets' orbital periods and mean distances from the sun.
	\begin{enumerate}[a.]
		\item 
		Plot the period time $T$ versus the mean distance $r$ using the observational data.
		% data table
		
		\item 
		Assuming a relationship of the form
		\[
		T = C r^a
		\]
		determine the parameters $C$ and $a$ by plotting $\ln{T}$ versus $\ln{r}$. Does the model seem reasonable? Try to formulate Kepler's third law.
		%Comment: just roughly estimate (using your computational software if you wish) the values of the parameters from the plots; do NOT apply any theory from later sections. Use the appropriate log transformation 
	\end{enumerate}
\end{enumerate}

\section*{Section 3.2}
\begin{enumerate}
	\item [2b.]
	Formulate the mathematical model that minimizes the largest deviation between the data and the line $y=ax+b$. Solve for the estimates of $a$ and $b$.
	\begin{table}[H]
		\centering
		\begin{tabular}{*{9}{c}} 
			\toprule
			$x$ & 29.1 & 48.2 & 72.7 & 92.9 & 118 & 140 & 165 & 199\\ \midrule
			$y$ & 0.0493 & 0.0821 & 0.123 & 0.154 & 0.197 & 0.234 & 0.274 & 0.328 \\
			\bottomrule
		\end{tabular}
	\end{table}

	\item 
	For the following data, formulate the mathematical model that minimizes the largest deviation between the data and the model $y = c_1 x^2 + c_2 x + c_3$. Solve for the estimates of $c_1$, $c_2$ and $c_3$.
	\begin{table}[H]
		\centering
		\begin{tabular}{*{6}{c}} 
			\toprule
			$x$ & 0.1 & 0.2 & 0.3 & 0.4 & 0.5 \\ \midrule
			$y$ & 0.06 & 0.12 & 0.36 & 0.65 & 0.95 \\
			\bottomrule
		\end{tabular}
	\end{table}
	%Comment: When solving the problems for Chebyshev Approximation Criterion, it is expected that your solution explicitly includes the linear program that has to be solved to apply CAC. After writing the linear program, you should solve it using any solver/ calculator.
\end{enumerate}

\end{document}

